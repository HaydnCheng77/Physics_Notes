\documentclass[english,a4paper,12pt]{report}
\usepackage{mypackage}

\title{Special Relativity}

\author{Haydn Cheng}

\date{\today}

\begin{document}
\maketitle
\tableofcontents

\chapter{Kinematics}

\section{Lorentz Transformation}

An event is something that an observer can specify with spatial and time coordiantes. For example somebody clap their hands, two spaceships crashes onto each other, or more abstractly a specific point at a given time. Something that is not an event is when the observer cannot specify the point in space where or the time when it occurs, such as a person staring at a pebble for one minute (because the time coordinate is not specified), or a train moving by an observer (because the train has spatial extent).

Lorentz transformation attempts to relate the coordinates of two different events\footnote{Since there is no preferred origin, an event can be described by arbitrary \(x \text { and } t\) values simply by shifiting the origin. Therefore there is no physical significance to compare the coordinates between two frames describing one event (except if the origin is specified but this already constitutes as an event).} observed in two different inertial frames (\(S \text { and } S'\)) which have a relative speed \(v\). That is, we want to find the constants \(A, B, C, D\) in the relations\footnote{We have assumed \(A, B, C \text { and } D\) are constants (does not depend on \(x, t, x' \text { or } t'\)), since all points in space are indistinguishable, the presence of dependency on absolute coordinates would mean that the absolute location in space (and not just the relative position) is important. In short, the absolute coordinate is arbitrarily defined based on the choice of the origin; only the change in coordinates is what we concern.}\footnote{We also assumed that \(\Delta x \text { and } \Delta t\) are linear functions of \(\Delta x' \text { and }  \Delta t'\). This can be justified by noting that any interval can be built up from a series of many infinitesimal intervals. But for infinitesimal intervals \(\Delta x' \text { and } \Delta t'\), any nonlinear terms such as \((\Delta t')^2\) are negligible. So if we add up all the infinitesimal intervals to obtain the given interval, we will be left with only the linear terms.}  

\begin{equation}
    \begin{aligned}
        \Delta x &= A \Delta x' + B \Delta t' \\
        \Delta t &= C \Delta t' + D \Delta x'. 
    \end{aligned}
\end{equation}

The four independent equations depending on four independent facts, which are 

\begin{enumerate}
    \item \(S'\) moves with velocity \(v\) with respect to \(S\).
    \item \(S\)  moves with velocity \(-v\) with respect to \(S'\).
    \item Time dilation (or length contraction) looks the same from either frame.
    \item A light pulse with speed \(c\) in \(S'\) also has speed \(c\)  in \(S\).
\end{enumerate}

The first two facts translate into

\begin{enumerate}
    \item \( \Delta x' = 0 \text { and } v = \Delta x/\Delta t \implies B /C = v \).
    \item \( \Delta x = 0 \text { and } -v = \Delta x' /\Delta t'\implies A = C \).  
\end{enumerate}

The third fact can be used by asking how fast does a person in \(S\) see a clock in \(S'\) ticks and the analogous question how fast does a person in \(S'\) see a clock in \(S\) ticks? 

The answer to the first question can be solved by letting \(\Delta x' = 0\), which gives \(\Delta t = \Delta t'\). The second question can be solved by letting \(\Delta x = 0\), which gives \(\Delta t' = \Delta t/(A-Dv) \). Comparing the two equations, we have \(D = (A-1 /A )/v \). 

The fourth fact can be used to say that if \(\Delta x' = c \Delta t'\), then \(\Delta x = c \Delta t\). Solving for \(A\)  gives \(A = \gamma \). 

Gathering all the constants found, we have the Lorentz transformation

\begin{equation}
    \begin{aligned}
    \Delta x &= \gamma (\Delta x' + v \Delta t') \\
    \Delta t &= \gamma (\Delta t' + v \Delta x').
    \end{aligned}
\end{equation}

Solving for \(\Delta x' \text { and }  \Delta t'\), we have the same form as above except \(v\) is replaced by \(-v\) (as expected, as there is no superior frame).   


If two events occur simultaneously in frame \(S\), then \(\Delta t = 0\), so \((\Delta x', \Delta t') = (\gamma \Delta x, -\gamma v \Delta x)\). Thus the two events does not occur simultaneously in frame \(S'\) and length is contracted.

\subsection{Time Dilation}

To prove that proper time (the time interval in which the events occur at the same place) is dilated we substitute \((\Delta x', \Delta t') = (0, \Delta t')\) into the Lorentz transformation which gives \((\Delta x,\Delta t) = (v\gamma \Delta t', \gamma \Delta t')\). 

\subsection{Length Contraction}

To prove that proper length (the length of an object in its rest frame) is contracted we substitute \((\Delta x', \Delta t) = (\Delta x', 0)\) into the Lorentz transformation which gives \((\Delta x, \Delta t') = (\Delta x' / \gamma , - v\Delta x' /c^2)\).  

The two relevant events here are that the measurement of the position of the head and the end of the object, which happens simultaneously in the unprimed frame \(S\) but not according to \(S'\).

\subsection{Velocity Addition}

Both longitudinal and transverse velocity addition can be easily done with Lorentz transformation, as

\begin{equation}
    \begin{aligned}
    v_{x}  &\equiv \frac{\Delta x}{\Delta t} = \frac{\Delta x' + v\Delta t'}{\Delta t' + v\Delta x'} = \frac{v_{x} ' + v}{1+v v_{x} '} \\
    v_{y}  &\equiv \frac{\Delta y}{\Delta t} = \frac{\Delta y'}{\gamma (\Delta t' + v\Delta x')} = \frac{v_{y} '}{\gamma (1 + v v_{x} ')},
    \end{aligned}
\end{equation}

where \(v\) is the relative speed of the two frames and \(v '\) is the velocity of the object observed in the primed frame.  

\subsection{Spacetime Interval}

As will be proved in later chapters, the spacetime interval (or invariant interval)

\begin{equation}
    (\Delta s) ^2 = (\Delta t)^2- (\Delta x)^2 - (\Delta y)^2 - (\Delta z)^2 
\end{equation}

is an invariant (does not depend on which reference frame you are measuring it from). 

There are 3 possible signs for \((\Delta s)^2\):

\begin{enumerate}
    \item \((\Delta s)^2 > 0 \implies \abs{x/t } < c \) (timelike separation): We can always find a frame \(S'\) in which the two events happen at the same place, since \(\Delta x' = \gamma (\Delta x - v\Delta t)\) can be equals to zero. This implies that it is possible for an object to travel from one event to the other.  
    
    In this case, the invariant interval \(\Delta s\) is the proper time \(\tau \), since \(\Delta x = \Delta y = \Delta z\) in the frame \(S'\) where the two events happen at the same place.    

    \item \((\Delta s)^2 < 0 \implies \abs{x/t} > c \) (spacelike separation): We can always find a frame \(S'\) in which the two events happen at the same time, since \(\Delta t' = \gamma (\Delta t - v\Delta x)\) can be equals to zero.
    
    Now the invariant interval \(\Delta s\) is no longer proper time, since there is no frame in which the two events happen in the same place. 
    
    The invariant interval \(\Delta s\) is now the negative of proper length, since \(\Delta t = 0\) in the frame \(S'\) where the two events happen at the same time.
    
    \item \((\Delta s)^2 = 0 \implies \abs{x/t } = c \) (lightlike separation): It is not possible to find a frame \(S'\) in which the two events happen at the same place or the same time. However, a photon emitted at one of the events will arrive at the other in every frame.       
\end{enumerate}

In a spacetime diagram, the third case corresponds to the events happen at the origin and on the light cone. Timelike events are the events at the origin and above or below the light cone. Spacelike separation are the events at the origin and at the left or right of the light cone.


\example{Exceeding the speed of light.}
{A laser beam is swepted through the space as it shines on the moon, the speed of the spot on the moon can be made arbitarily large as it is directly proportional to the distance from the moon to the Earth, why does it not violate special relativity?}
{Each photon travels independently at the speed of light. The speed of the spot does not bear any physical significance since it is defined based on uncorrelated events (each photon hitting the moon), and thus cannot be used to transfer any information.} 


\section{Doppler Effect}

\subsection{Classical Doppler Effect}

The classical limit is when the source and observer speeds, \(v_{s}\text { and } v_{o}  \) respectively, are much smaller than the speed of light. Note, however, that the speed of the wave \(v\) can be close to the speed of light \(c\), \textit{i.e.,} \(v_{o} \ll c \text { and } v_{s} \ll c  \). 

\subsubsection{One-Dimensional}

In one-dimensional classical case, the doppler effect formula is trivially given by

\begin{equation}
    f_{o} = \frac{v \pm v_{o} }{v \pm v_{s} } f_{s},  
\end{equation}

where the \(\pm \) sign can be deduced from common sense.

\subsubsection{Two-Dimensional}

In two-dimensional classical case, the doppler effect formula is modifed to only take into account only the relevant components of velocities

\begin{equation}
    f_{s} = \frac{v \pm v_{o} \cos \theta _{o}  }{v \pm v_{s} \cos \theta _{s} } f_{o} ,  
\end{equation}

where \(\theta _{s} \) is the angle between the angle between the sound and the source and \(\theta _{o} \) is the angle between the sound and the observer. 

\onefig{doppler}{scale=0.3} 

From \cref{doppler} we have 

\begin{equation}
    \abs{\vb{x} } = ct ~\text { and }~ \abs{\vb{x} + \vb{v} _{0}\delta t_0 - \vb{v} _{s} \delta t_{s}   }^2 = c^2(t+\delta t_0 -\delta t_{s} )^2,  
\end{equation}

which implies 

\begin{equation}
    \vb{x} \cdot \vb{v} _{0}\delta t_o - \vb{x} \cdot \vb{v} _{s} \delta t_{o} = \abs{\vb{x} } c(\delta t_o - \delta t_{s} ),    
\end{equation}

so we have 

\begin{equation}
    f_{o} = \frac{\delta t_s }{\delta t_{o} } f_{s} = \frac{v-\vb{v} _{o} \cdot \vu{x}   }{v-\vb{v} _{s}\cdot \vu{x}  } f_{s}.  
\end{equation}

Note that for relativistic case this reduces to 

\begin{equation}
    f_{o} = \frac{\delta t_{s} }{\delta \tau _{o} } f_{s} = \frac{\gamma _{s}\delta t_{s}  }{\gamma _{o} \delta t_{o}  } f_{s}.    
\end{equation}

\subsection{Relativistic Doppler Effect}

The relativistic limit is that the relative speed of the source and the observer \(v\) is comparable to the speed of light \(c\), \textit{i.e.,} \(v ~ c\). 

\subsubsection{One-Dimensional}

In one-dimensional relativistic case, the doppler effect formula is given by 

\begin{equation}
    f_{o} = \sqrt{\frac{c\pm v}{c \mp v} } f_{s} ,  
\end{equation}

where the \(\pm \) sign can be deduced from common sense.

The formula is easiest proven in the observer's frame (see \cref{obv}), where from the one-dimensional classical doppler effect formula and time dilation formula we have 

\begin{equation}
    f_{o} = \frac{c}{c+v} f_{s}', \quad f_{s}' = \frac{f_{s} }{\gamma } \implies f_{o} = \sqrt{\frac{1-\beta }{1+\beta } } f_{s}.   
\end{equation}

where \(f_{s} '\) is the frequency of the source observed by the observer, while \(f_{s} \) is the frequency of the source in the source's rest frame. 

One can also proof the formula in the source's frame (see \cref{source}), using the Lorentz transformation 

\begin{equation}
    (\Delta x,\Delta t) = \left( \frac{vc}{c-v} \frac{1}{f_{s} }, \frac{c}{c-v} \frac{1}{f_{s} }     \right) \implies (\Delta x', \Delta t') = \left( 0, \sqrt{\frac{1+\beta }{1-\beta } } \frac{1}{f_{s} }  \right).
\end{equation}

\onefig{obv}{scale=0.3} 

\onefig{source}{scale=0.3} 


We also note that when \(v \ll c\) the formula is reduced to 

\begin{equation}
    f_{o} \approx \frac{c+v}{c} f_{s} \approx \frac{v}{c+v} f_{s},     
\end{equation}

as it should by the correspondence principle.  
























\chapter{4-Vectors and Dynamics}

\section{Definition and Examples of 4-Vectors}

\begin{definition}
A 4-tuple, \(A = (A_0 , A_1 , A_2 , A_3 )\) is called a ``4-vector'' if the \(A_{i} \) transform between frames in the same way that \((\Delta t, \Delta x, \Delta y, \Delta z)\) do. In other words, \(A\) is a 4-vector if it transforms according to the Lorentz transformation

\begin{equation}
    \begin{pmatrix}
         A_0  \\
         A_1  \\
         A_2  \\
         A_3  \\
    \end{pmatrix} = \gamma \begin{pmatrix}
        1  & v & 0 & 0  \\
        v & 1 & 0 & 0  \\
        0 & 0 & 1 & 0  \\
        0 & 0 & 0 & 0  \\
    \end{pmatrix} \begin{pmatrix}
         A_1 ' \\
         A_2 ' \\
         A_3 ' \\
         A_4 ' \\
    \end{pmatrix}.
\end{equation}
\end{definition}

The displacement 4-vector between two events that are infinitesimally close together is 

\begin{equation}
    dS \equiv (dt,dx,dy,dz)
\end{equation}

Dividing the displacement 4-vector with the proper time \(d \tau = \frac{dt}{\gamma } \) (which is an invariant since it is independent of the fram in which it is measured), we obtain 

\begin{equation}
    V \equiv  \frac{dS}{d\tau } = \gamma \left(1, \frac{dx}{dt}, \frac{dy}{dt}, \frac{dz}{dt}\right) = \gamma (1, \vb{v} ).
\end{equation}

We can then take the derivative of the velocity 4-vector with respect to \(\tau \). The result is a four vector because the numerator is just an infinitesimal change of the displacement 4-vector (which is also a 4-vector due to linearity) and the denominator is an invariant. So

\begin{equation}
    A \equiv  \frac{dV}{d\tau } = \gamma \left(\frac{d\gamma }{dt}, \frac{d(\gamma \vb{v} )}{dt}\right) = (\gamma ^{4}v \dot{v} , \gamma ^{4} v \dot{v} \vb{v} + \gamma ^2 \vb{a})
\end{equation}

Multiply the velocity 4-vector by another invariant, mass \(m\), we get the energy-momentum 4-vector

\begin{equation}
    P \equiv mV = \gamma m(1,v) = (E, \vb{p} ). 
\end{equation}

The force 4-vector is then the derivative of the momentum 4-vector with respect to \(\tau \). This gives

\begin{equation}
    F \equiv \frac{dP}{d\tau } = \gamma \left(\frac{dE}{dt}, \vb{f}\right),
\end{equation}

where \(\vb{f} \equiv d(\gamma m\vb{v} )/dt \) is the usual 3-force defined as the rate of change of momentum.  

In terms of acceleration 4-vector, we can write 

\begin{equation}
    F = mA = m(\gamma ^{4}v \dot{v} , \gamma ^{4} v \dot{v} \vb{v}  + \gamma ^2 \vb{a}  ) = m(\gamma ^{4} v_{x} a_{x}, \gamma ^{4} a_{x}, \gamma ^2 a_{y}, \gamma ^2 a_{z}),
\end{equation}

if \(\vb{v} = v_{x}\vu{x}  \). Combining the two forms of \(F\), we see that the 3-force is

\begin{equation}
    \vb{f} = m(\gamma ^3 a_{x}, \gamma a_{y}, \gamma a_{z}).
\end{equation}

\section{Properties of 4-Vectors}

If \(A\) and \(B\) are 4-vectors, then \(C \equiv aA + bB = \) is also a 4-vector, this is due to the linearity of Lorentz transformation. For example, the first component of \(C\) is

\begin{equation}
    C_0 \equiv aA_0 + bB_0  = a(A_0 ' + v A_1 ') + b(B_0 ' + vB_1 ') = (aA'_{0} + b B'_{0}) + v(aA'_{1} + bB'_{1}  ) = C'_{0} + vC'_{1}  
\end{equation}

which is the correct transformation for the first component of \(C\) under Lorentz transformation.

The inner product of two 4-vectors \(A \text { and } B\) are defined as 

\begin{equation}
    \sum_{\mu} A ^{\mu }B _{\mu } \equiv A ^{\mu } B _{\mu}  \equiv  \vb{A}  \cdot B  \equiv A_0 B_0 - A_1 B_1  - A_2 B_2 - A_3 B_3,
\end{equation}

where \(\vb{P} _{\mu } = (\text{diag}(1,-1,-1,-1) \vb{p} ^{\mu }  )^T \).  

It can be shown (by direct substitution) that the inner product is an invariant. This invariance under Lorentz Transformation (by multiplying the Lorentz matrix) is analogous to the invariance of the scalar (dot) product \(\vb{A} \cdot \vb{B} = \abs{A}\abs{B}\cos \theta   \) under rotations (by multiplying the rotational matrix) in 3-space (since neither \(\abs{\vb{A} }, \abs{\vb{B} }   \) nor \(\theta \) is affected by rotation ). 

The norm of a 4-vector is then defined as the square root of the inner product of \(A\) with itself:

\begin{equation}
    \abs{A}^2 \equiv A \cdot A \equiv A_0 A_0 - A_1 A_1 -A_2 A_2 -A_3 A_3  = A_0 ^2 - \abs{\vb{A} }^2.  
\end{equation}

The invariance of the norm \(\sqrt{A \cdot A} \) is analogous to the invariance of the norm \(\sqrt{\vb{A} \cdot \vb{A} } \equiv \abs{\vb{A} } \) for rotations in 3-space. 

If a certain one of the components of a 4-vector is zero in every frame, then all four components are zero in every frame. This is analgous to how if someone comes along and says the she has a vector in 3-space that has no \(x\) component, no matter how you rotate the axes, then you would certainly say that the vector must be the zero vector. The situation in Lorentzian 4-sapce is the same, because all of the coordinates get intertwined with each other in the Lorentz (and rotation) transformations.
\section{Force and Acceleration}

In frame \(S'\) of a given particle travelling at speed \(v\) in the \(x\) direction relative to the lab frame \(S\), we have the force 4-vector

\begin{equation}
    F' = \gamma (\frac{dE'}{dt}, \vb{f} ' ) = \gamma ( \frac{dm}{dt}, \vb{f} ' ) = (0, \vb{f} ')
\end{equation}

Transfomring \(F'\) into \(F\), we get

\begin{equation}
    \begin{aligned}
        F_0 &= \gamma \frac{dE}{dt} = \gamma (F_0 ' + v F_1 ') = \gamma v f_{x}' , \\
        F_1 &= \gamma f_{x} =  \gamma (F_1 ' + v F_0 ') = \gamma f_{x}', \\
        F_2 &= \gamma f_{y} = F_2 ' =  f_{y}' , \\
        F_2 &= \gamma f_{z} = F_2 ' =  f_{z}' . \\
    \end{aligned}
\end{equation}

Thus we get 

\begin{equation}
        \frac{dE}{dt} = vf_{x}',  ~~
        f_{x} = f_{x}', ~~
        f_{y} = \frac{f_{y}' }{\gamma }, ~~
        f_{z} = \frac{f_{z}' }{\gamma }.   
\end{equation}

The acceleration 4-vector, on the other hand, is 

\begin{equation}
    A' = (\gamma ^{4}v' \dot{v'} , \gamma ^{4} v' \dot{v'} \vb{v'} + \gamma ^2 \vb{a'}) = (0,\vb{a} '),
\end{equation}

Transforming \(A'\) into \(A\), we get 

\begin{equation}
    \begin{aligned}
        A_0 &= \gamma ^{4}va_{x} = \gamma (A_0 ' + vA_1 ') = \gamma v a_{x}', \\
        A_1 &= \gamma ^{4} a_{x} = \gamma (A_0 ' + vA_0 ' ) = \gamma a_{x}', \\
        A_2 &= \gamma ^2 a_{y} = A_2 ' = a_{y}', \\
        A_3 &= \gamma ^2 a_{z} = A_3 ' = a_{z}'.           
    \end{aligned}
\end{equation}

Thus we get

\begin{equation}
    a_{x} = \frac{a_{x}' }{\gamma ^3 }, ~~ a_{y} = \frac{a_{y}' }{\gamma ^2}, ~~ a_{z} = \frac{a_{z}' }{\gamma ^2}.      
\end{equation}

\example{Acceleration for Circular Motion}
{A particle moves with constant speed \(v\) around the circle \(x^2+y^2=r^2, ~z=0\), in the lab frame. At the instant the particle crosses the negative \(y\) axis, find the 3-acceleration and 4-acceleration in both the lab frame and the instantaneous inertial frame of the particle. }
{The 3-acceleration in \(S\) is simply

\begin{equation}
    \vb{a} = \left(0,\frac{v^2}{r}, 0 \right).
\end{equation}

There is nothing fancy going on here; the standard nonrelativistic proof of the centripetal acceleration works just fine again in the relativistic case. 

The 4-accleration in \(S\) is then 

\begin{equation}
    A = \left(0,0,\gamma ^2\frac{v^2}{r}, 0 \right).
\end{equation}

Transforming to \(S'\), we have 

\begin{equation}
    A' = \left(0,0, \gamma ^2 \frac{v^2}{r}, 0 \right).
\end{equation}

So the 3-acceleration in \(S'\) is then the space part of \(A'\)

\begin{equation}
    \vb{a} ' = \left(0,\gamma ^2\frac{v^2}{r}, 0 \right).
\end{equation}

Of course, we can arrive at this result with a simple time-dilation argument, since

\begin{equation}
    a_{y}' = \frac{d^2y'}{dt'^2} = \frac{d^2y}{(dt /\gamma ) ^2} = \gamma ^2 a_{y} = \gamma ^2 \frac{v^2}{r}.     
\end{equation}
}

Note that in special relativity, if a statement has any chance of being true in all frames, it must involve only 4-vectors. Consider a 4-vector equation \(A = B\) that is true in frame \(S\). Aplying Lorentz transformation on both sides, we get \(A' = B'\) so the law is therefore also true in frame \(S'\). However, \(\vb{f} = m\vb{a} \) cannot be a physical law, since the two sides of this equation transform differently when going from one frame to another, so the statement cannot be true in all frames. 

All of this is exactly analogous to the situation in 3-dimensional space. In Newtonian mechanics, \(\vb{f}  = m\vb{a} \) is a possible law (and indeed a physcal law), because both sides are 3-vectors. But \(\vb{f} = m(2a_{x}, a_{y},a_{z})\) has no chance of being a physical law, because the right hand side is not a 3-vector; it depends on which axis you label as the \(x\) axis.

Physcial laws may also take the form of scalar equations, such as \(P \cdot P = m^2\). A scalar is by definition a quantity that is frame independent. So if a scalar statement is true in one inertial frame, then it is true in all inertial frames.  

\section{Collisions}

From the energy-momentum 4-vector we see that the momentum and energy of a particle governed by special relativity are

\begin{equation}
    \vb{p}  = \gamma m\vb{v} \text { and } E = \gamma m.
\end{equation}

Taylor expanding \(\gamma \) , we get

\begin{equation}
    \begin{aligned}
    \vb{p} &= (1 + \mathcal{O}(v^2)) m\vb{v} = m\vb{v} + \mathcal{O}(v^3 ) \\
    E &= (1 + \frac{1}{2}v^2 + \mathcal{O}(v^{4} ) ) m = m + \frac{1}{2}mv^2 + \mathcal{O}(v^{4} ), 
    \end{aligned}
\end{equation}

which, according to correspondence principle, must give the same answer in Newtonian physics. 

Note that the energy of a particle even in Newtonian physics consists of the rest mass energy term \(m\).  

In an elastic collision, no kinetic energy is transformed into other forms of energy, so the mass is conserved (To not introduce another degree of freedom, we assume that individual masses are conserved, not just the sum of them, so particles cannot transfer their internal energy to one another), so we can neglect writing down the rest mass energies when applying the law of conservation of energy.

An inelastic collision, on the other hand, converts kinetic energy to internal energy (e.g., thermal energy) which increases the masses of the particles\footnote{All internal energy such as thermal, rotational and potential energy contributes to the rest mass of an object. In fact, the vast majority of the mass of an atom is due to the internal energy between quarks that make up the nucleus rather than the rest mass of the quarks themselves.}, so the mass is not conserved and thus the Newtonian kinetic energy \(mv^2 /2\) is not conserved. 

Similar to the relation \(E = p^2/2m \text { or } p = \sqrt{2mE}  \) in Newtonian physics, we can relate the energy and the momentum of a particle by this ``very-important-relationship''

\begin{equation}\label{veryimportantrelationship} 
    E^2 = p^2 + m^2.
\end{equation}

Whenver we know two of the three quantities \(E,p \text { and }  m\), this eqution gives you the third.

There is no need to use 4-vectors to solve collision problems, they can be always be solved solely by writing down the energy and momentum conservation laws, as well as the ``very-important-relationship'' but they often prove to be useful bookeeping tools, since the norm of the energy-momentum 4-vector of a particle (note that this does not work for a collection of particles) is simply the mass of the particle: 

\begin{equation}
    P \cdot P = E^2 - \vb{p} ^2 = m^2.
\end{equation}

The power of writing energy and momentum of a particle in a single energy-momentum 4-vector can be illustrated by the example below:

\example{Decay of a Particle.}
{A particle with mass \(M\) and energy \(E\) decays into two identical particles. In the lab frame, one of them is emitted at a \(90 ^\circ \) angle. What are the energy of the created particles? }
{Let \(P, P_1 \text { and } P_2 \) be the energy-momentum 4-vectors of the decayed particle, the particle emitted at \(90 ^\circ \) angle, and the remaining particle respectively. 

By conservation of energy and momentum,

\begin{equation}
    \begin{aligned}
        P &= P_1 + P_2 \\
        P-P_1  &= P_2 \\
        (P - P_1 )(P - P_1 ) &= P_2 \cdot P_2 \\
        P^2 - 2 P \cdot  P_1  + P_1 ^2 &= P_2 ^2 \\
        M^2 - 2E E_1  + m^2 &= m^2 \\
        E_1  & = \frac{M^2}{2E}. 
    \end{aligned}
\end{equation}

Again, the energy-momentum vector introduce no new physics, it simply write the conservation laws in a way that help simplify the algebra using the ``norm squared equals mass'' property.
} 

The only non-trivial property of the energy-momentum vector is that since the sum of the energy-momentum 4-vectors of a collection of particles \(\sum_{i=1}^{n} P\)  is also a 4-vector due to linearity, the norm squared of this 4-vector

\begin{equation}
    \left( \sum_{}^{} P \right)^2 = \left( \sum_{}^{} E \right) ^2 - \left( \sum_{}^{} \vb{p}   \right) ^2 
\end{equation}

is also an invariant. Note that the sums are taken before squaring in the above equation. If we sqaure before the addition it would simply give the sum of the squares of the masses (which of course is also an invariant but is trivial why).

\example{Threshold Kinetic Energy.}
{What is the threshold kinectic energy in lab frame for production of an electron positron pair by collision of an incident electrron with mass \(m_{e} \) with a stationary electron? }
{Since \(\sum \vb{p}_{\text{c.m.}}  =0 \text { and } \sum E_{\text{c.m.}} = \sum \gamma _{\text{c.m.}}m  \), the kinetic energy in the center of mass frame is minimized if \(\gamma _{i} = 0\) for all \(i\). 

Using the subscript ``lab'' and ``c.m.'' to distinguish the lab frame and the center of mass frame; primed and unprimed to distinguish the quanitity before or after collision, we have 4 equal quantities:

The norm squared of the sum of the energy-momentum 4-vectors of all the particles in the center of mass frame after the collision as 

\begin{equation}
    \left( \sum P_{\text{c.m.}}' \right)^2 = \left( \sum_{}^{} E_{\text{c.m.}}'  \right) ^2 - \left( \sum_{}^{} \vb{p}_{\text{c.m.}}'    \right) ^2  = \left( \sum m \right)^2 = 4 m_{e}^2,
\end{equation}

which must be equals to the normed squared of the sum of the energy-momentum 4-vectors of all the particles in the lab frame after the collsion due to 4-vector property

\begin{equation}
    \left(\sum P_{\text{lab} }' \right)^2 = \left(\sum E_{\text{lab} }'\right)^2 - \left( \sum \vb{p} _{\text{lab} }' \right)^2 \text{(not important)},
\end{equation}

which must be equals to the normed squared of the sum of the energy-momentum 4-vectors of all the particles in the lab frame before the collsion due to energy and momentum conservation

\begin{equation}
    \left(\sum P_{\text{lab} } \right)^2 = \left(\sum E_{\text{lab} }\right)^2 - \left( \sum \vb{p} _{\text{lab} } \right)^2 = (E+m_{e} )^2 - p^2 = (E+m_{e} )^2 - E^2 + m_{e}^2. 
\end{equation}

which must be equals to the normed squared of the sum of the energy-momentum 4-vectors of all the particles in the center of mass frame before the collsion due to 4-vector property

\begin{equation}
    \left(\sum P_{\text{c.m.} } \right)^2 = \left(\sum E_{\text{c.m.} }\right)^2 - \left( \sum \vb{p} _{\text{c.m.} } \right)^2 \text{(not important)}. 
\end{equation}

Alternatively, we can use the lorentz transformation of the energy-momentum 4-vectors. Here there is only one \(\gamma ~(\text {or } \beta )\) is important, since the \(\gamma \) of the particle in the center of mass frame is equals to the \(\gamma \) of the center of mass frame relative to the lab frame, as one of the electron is originally at rest and the masses of the two electrons are equal.

In the center of mass frame before the collision, we have 

\begin{equation}
    \sum P_{\text{c.m.} }  = \begin{pmatrix}
         2\gamma  m_{e}  \\
         0 \\
    \end{pmatrix}.
\end{equation}

Boosting into the lab frame before the collision, we have

\begin{equation}
    \sum P_{\text{lab} } = \begin{pmatrix}
         2 m_{e} \gamma^2   \\
         -2 m_{e} \gamma\beta  \\
    \end{pmatrix}.
\end{equation}

By conservation of energy and momentum, we have

\begin{equation}
    \sum P_{\text{lab}}' = \begin{pmatrix}
         2 m_{e} \gamma ^2 \\
         -2 m_{e} \gamma \beta  \\
    \end{pmatrix}. 
\end{equation}

Boosting into the center of mass frame after the collision, we have

\begin{equation}
    \sum P_{\text{c.m.} }' = \begin{pmatrix}
         2 m_{e} \gamma  \\
         0 \\
    \end{pmatrix}.
\end{equation}

However, we also know that in the center of mass frame all the particles are at rest, so 

\begin{equation}
    \sum P_{\text{c.m.} }' = \begin{pmatrix}
         4 m_{e}  \\
         0 \\
    \end{pmatrix}.
\end{equation}

Threrfore we have \(\gamma = 2\).

Both methods yields \(E = 7m_{e} \implies K.E. = 6m_{e} \). 
}

\example{Photon Emission.}
{Show that a free electron moving in a vacuum cannot emit a single photon.}
{An incomplete approach would be to write down the energy and momentum conservation laws in the lab frame, since the the direction of photon emission need not be the same with the direction in which the electron is originally moving. Working in the center of mass frame, we have

\begin{equation}
    \begin{pmatrix}
         m_{e}  \\
         0  \\
    \end{pmatrix} = \begin{pmatrix}
         \gamma m_{e}  \\
         \gamma \beta m_{e}  \\
    \end{pmatrix} + \begin{pmatrix}
         E_{p}  \\
         E_{p}  \\
    \end{pmatrix}. 
\end{equation}

The only solution is that \(\gamma = 1, E_{p}=0, \beta = 0\), which means that there is no emission of photon.

} 

\example{Decay of an Atom.}
{An atom in an excited state of energy \(Q_0 \) (as measured in its rest frame) above the ground state is moving with velocity \(v\) as observed from the lab frame. The atom decays to its ground state by emitting a photon of energy \(Q\) and comes completely at rest. Find \(Q\) if the mass of the atom is \(M\).}
{Comparing the norm squared of the energy-momentum 4-vector between the center of mass frame before the decay and the lab frame after the decay, we have

\begin{equation}
    \left( \sum P \right)^2 = (Q_0 + M)^2 = (M + Q)^2 - Q^2 \implies Q = Q_0 \left( 1 + \frac{Q_0 }{2Mc^2} \right).
\end{equation}

Alternatively, we have

\begin{equation}
    \begin{pmatrix}
         Q_0 + M \\
         0 \\
    \end{pmatrix}_{\text{c.m.} }  \rightarrow \begin{pmatrix}
         \gamma (Q_0 + M) \\
         v \gamma (Q_0 + M) \\
    \end{pmatrix}_{\text{lab} } = \begin{pmatrix}
         Q + M \\
         Q \\
    \end{pmatrix}'_{\text{lab} }.
\end{equation}

Therfore, we have \(\gamma = (Q+M) /(Q_0 +M) \text { and } v\gamma = Q /(Q_0 +M)\). Using the identity \(\gamma ^2- \beta ^2\gamma ^2 = 1\), we get the same result \(Q = Q_0 \left( 1 + Q_0 /2Mc^2 \right)\).

The important point here to note is the the internal energy \(Q_0 \) is added to \(M\) are transformed together, thus internal energy is equivalent to mass.  
} 

\example{Meson-Photoproduction.}
{\begin{enumerate}
    \item Consider a group of \(N\) particle, where the \(i^{\text{th}} \) particle has energy \(E_{i} \) in the laboratory frame. Starting from the Lorentz transformations of energy and momentum show that the Lorentz factor of the center of mass frame, \(\gamma _{\text{c.m.} } \), can be written as 
    
    \begin{equation}
        \gamma _{\text{c.m.} } = \frac{\sum_{i=1}^{N} E_{i} }{E_{\text{c.m.} } }. 
    \end{equation}
    
    \item Consider a photon which collides with a proton at rest in the laboratory frame. The collision produces a neutral \(\pi \)  meson and a proton via 
    
    \begin{equation}
        \gamma + p \to \pi + p.
    \end{equation}
    
    Find the threshold photon energy for this reaction to occur.

    \item A \(\pi \) meson is produced at threhold via the interaction given in (b), and subsequently decays to two photons. One photon travels in a directino parallel, and the other anti-parallel, to the initial line of light of the \(\pi \) meson in the laboratory frame. Calculate the energy of each photon in the laboratory frame.     
\end{enumerate}
~
}
{\begin{enumerate}
    \item The Lorentz transformation of energy and momentum are 
    
    \begin{equation}
        \begin{cases}
            E' &= \gamma (E-vp),\\
            p' &= \gamma (p-vE).
        \end{cases}
    \end{equation}
    
    Upon summation we have 

    \begin{equation}
        \begin{cases}
            \sum_{i=1}^{N} E' &= E_{\text{c.m.} } = \gamma \left(\sum_{i=1}^{N} E_{i} - v\sum_{i=1}^{N} p_{i} \right),\\
            \sum_{i=1}^{N} p' &= p_{\text{c.m.} } = 0 = \gamma \left( \sum_{i=1}^{N} p_{i} - v\sum_{i=1}^{N} E_{i}   \right).
        \end{cases}
    \end{equation}
    
    Eliminating \(\sum_{i=1}^{N} p_{i} \) we get 
    
    \begin{equation}
        \gamma _{\text{c.m.} } = \frac{\sum_{i=1}^{N} E_{i} }{E_{\text{c.m.} } }. 
    \end{equation}
    
    This is incidentally a very useful result, when the usual forula 

    \begin{equation}
        \gamma _{\text{c.m.} } = \frac{\sum_{i=1}^{N} p_{i} }{\sum_{i=1}^{N} E_{i}  } } 
    \end{equation}
    
    does not work.
    
    \item From the energy-momentum invariant we get 
    
    \begin{equation}
        (m_{\pi} + m_{p} )^2 = (E_{\gamma } + m_{p}  )^2 - E_{\gamma }^2 \implies E_{\gamma } = m_{\pi}\left( 1+\frac{m_{\pi} }{2m_{p} }  \right).
    \end{equation}
    
    \item We first notice that the Lorentz factor of the \(\pi \) meson is the same as the Lorentz factor of the center of mass, since the \(\pi \) meson is stationary in the center of mass frame after the threshold collision, so
    
    \begin{equation}
        \gamma _{\pi } = \gamma _{\text{c.m.} } = \frac{\sum_{i=1}^{N}E_{i}  }{E_{\text{c.m.} } }=  \frac{E_{\gamma } + m_{p}  }{m_{\pi} + m_{p}  }.    
    \end{equation}
    
    In the \(\pi \) meson's frame, the momentum and energy of the photons are 
    
    \begin{equation}
        p_{\gamma } = \pm \frac{m_{\pi} }{2} ~\text { and }~ E_{\gamma } = \frac{m_{\pi} }{2}.    
    \end{equation}
    
    We can now simply boost to lab frame to obtain the final answer

    \begin{equation}
        E_{\text{lab} } = \gamma _{\pi } (E_{\gamma } + v_{\pi} p_{\gamma }   ) = \SI{77.2}{\MeV } ~\text { or }~ \SI{59.0}{\MeV }.  
    \end{equation}
\end{enumerate}
~
} 


\example{Collision.}
{A ball of rest mass \(M\) and energy \(E\) collides elastically with a stationary ball of rest mass \(m\), such that the direction of motion is along the line joining the centers of the two balls. Find the final energy of \(M\).}
{From the conservation of the energy-momentum 4-vectors, we have

\begin{equation}
    \begin{pmatrix}
         E\\\
         \sqrt{E^2-M^2}  \\
    \end{pmatrix} + \begin{pmatrix}
         m \\
         0 \\
    \end{pmatrix} = P_{m} + \begin{pmatrix}
         E' \\
         \sqrt{E'^2-M^2}  \\
    \end{pmatrix}.
\end{equation}

Isolating \(P_{m} \) on one side and squaring both sides, 

\begin{equation}
    \begin{aligned} 
    2M^2 + m^2 + 2(mE) - 2 (EE' - \sqrt{E^2-M^2}\sqrt{E'^2-M^2}  ) -2(mE') &= m^2 \\
    (M^2 + mE - mE' - EE')^2 +(E^2-M^2)(E'^2-M^2) &=0  \\
    m^2 (E-E')^2 + 2M^2m(E-E') - 2M^2EE' - 2m(E-E')EE' + M^2E^2 + M^2E'^2 &=0 \\ 
    m^2(E-E')^2 + 2M^2m(E-E') - 2m(E-E')EE' +M^2(E-E')^2 &= 0\\
    E' = \frac{2mM^2+E(m^2+M^2)}{2Em+m^2+M^2}, 
    \end{aligned} 
\end{equation}

where we have divided the whole equation by \(E-E'\), indicating that \(E= E'\) is a trivial solution. 
} 


\section{Force}

The Newton's law in special relativity is 

\begin{equation}
    F = \frac{dp}{dt} = \frac{d(\gamma mv)}{dt} = m(\dot{\gamma }v + \gamma \dot{v}  ) = m(\gamma ^3 vav + \gamma a) = ma\gamma (\gamma ^2v^2 + 1) = \gamma ^3 ma,
\end{equation}

for one-dimensional case while the work energy theorem is 

\begin{equation}
    \int_{x_1 }^{x_2 } Fdx = \int_{x_1 }^{x_2 } (\gamma ^3 ma) dx = m \int_{v_1 }^{v_2 } \gamma ^3 vdv = m \int_{\gamma _{1} }^{\gamma _{2} } d\gamma = E_2 - E_1 = \Delta E.      
\end{equation}

which is the same as in nonrelativistic physics.

The Newtons's law can be easily generalized into two dimensions by 

\begin{equation}
    F = \frac{d\vb{p} }{dt} = \frac{d(\gamma m\vb{v} )}{dt} = \frac{d}{dt} \left(\frac{m(v_{x}, v_{y}  )}{\sqrt{1 - v_{x}^2 - v_{y}^2  } }\right) = m(\gamma ^3 a_{x}, \gamma a_{y}  ).    
\end{equation}

The asymmetric arises from the fact that \(v_{x} = v \text { and }  v_{y} = 0 \) at the start by default.

\example{Bead on a Rod}
{A spring with a tension has one end attached to the end of a rod, and the other end attached to a bead that is constrained to move along the rod. The rod makes an angle \(\theta '\) with respec to the \(x'\) axis and is fixed at rest in the \(s'\) frame (see \cref{beadonarod1}). Right after the bead is release, find the directions of the rod, the acceleration of the bead and the force on the bead in frame \(S\).}
{The horizontal span of the rod is decreased by a factor \(\gamma \) due to length contraction while the vertical span is unchanged. So we have \(\tan \theta = \gamma \tan \theta '\). 

The acceleration, on the other hand, must point along the rod, simply because the bead always lies on the rod (and the rod itself does not accelerate). 

The force, however, does not point along the rod. This is because the \(y\) component of the force on the bead is decreased by a factor of \(\gamma \) while the \(x\) component is unchanged. So \(F_{y}/F_{x}   \) is smaller than \(F_{y}'/F_{x}'   \) by a factor \(\gamma \). So we have \(\tan \phi = \tan \theta ' /\gamma  \). 

To double check the direction of acceleration, 

\begin{equation}
    \frac{a_{y} }{a_{x} } = \gamma ^2 \frac{F_{y} }{F_{x} } = \gamma ^2\frac{F_{y}' /\gamma    }{F_{x}' }  = \gamma \tan \theta ' = \tan \theta .  
\end{equation}

Note that the rod does not exert a force of constraint. The bead does not need to touch the rod in \(S'\) so it does not need to touch it in \(S\). \(\vb{F} \) simply does not have to be collinear with \(\vb{a} \) in relativistic physics.  

The situation in frame \(S\) is shown in \cref{beadonarod2}.

} 
\twofig{beadonarod1}{width=\textwidth}{beadonarod2}{width=\textwidth}{beadonarod} 
































\end{document}
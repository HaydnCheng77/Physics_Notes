\documentclass[english,a4paper,12pt]{report}
\usepackage{mypackage}

\title{Optics}

\author{Haydn Cheng}

\date{\today}

\begin{document}
\maketitle
\tableofcontents
    
\chapter{Imaging Systems}

Imaging systems are devices that aim to produce an image of an object. For this to happen we need that all the rays of light leaving from a single point \(A\) in the object arrive at a single point \(A'\) in the image, whatever their point of entry \(M\) in the device. 

Since Fermat's principle tells us that light always takes the fastest path from \(A\) to \(A'\), this means that the optical path of all rays emerging from \(A\) and arriving at \(A'\) must be the same and equal to an extrema.   

In almost all of geometrical optics problems, we employ the paraxial approximation, which assumes that all light rays are nearly parallel to the optical axis, which implies that the angles formed by the light rays with the optical axis is always small enough that the small angle approximations are valid.

\section{Fundamentals}

\subsection{Sign Conventions}

When discussing spherical mirrors and spherical surfaces, we will see that we can subdivide each of them cases. However, only one equation is needed for one case. This is because we adopt the following sign convention to embbed then negative signs into distances:

\begin{enumerate}
    \item The object distance \(u\) is positive if the object point is one the same side as the incident light.
    \item The image distance \(v\) is positive if the image point is one the same side as the outgoing light (this implies a real object).
    \item The radius \(R\) is positive if the center of curvature is on the same side as the outgoing light.
\end{enumerate}

An easy way to remember to remember the convention of the object distance and the image distance is that they are positive if the object point and the image point are where they are supposed to be at.

\subsection{Spherical Mirrors} \label{spheremirsec} 

Referring to \cref{spheremir}\footnote{Note that the arrows drawn in the figure may be reversed due to light's reversibility.}, there are two possibilities in which light can reflect off a spherical mirror.

\onefig{spheremir}{scale=0.3} 

The first possibility, labelled as path 1, is that the object (at this stage we assume that it is real) is situated inside the sphere, which creates a real object. The second possibility, labelled as path 2, is that the object is situated outside the sphere, which would create a virtual object.

However, using the sign convention defined above, we can generalized the two equations governing the two cases as 

\begin{equation}
    \frac{1}{u} + \frac{1}{v} = \frac{2}{R} \equiv \frac{1}{f}.
\end{equation}

The quanitity \(\frac{1}{u} + \frac{1}{v}\) of a imaging system is a constant, then the imaging system will  
The case where the object distance \(u\) is negative, which only occurs when there are more than two imaging systems, can be regarded as if \(u\) is real but \(v \text { and } R\) are negative, which corresponds to switching between the first case and the second case mentioned above. Therefore, it is equivalent as if there is a real object situated at the opposite side of the mirror with the same distance from the mirror. This implies that we can simply treat the image point of the first imaging system as the object point of the second imaging system.

\subsection{Spherical Surfaces}

Referring to \cref{spheresur}\footnote{See footnote under \cref{spheremirsec}.}, there are two possibilities\footnote{There are actually four, the other two being if the object is situated at the opposite side of the spherical surface compared to what is shown above. But we have already shown in \cref{spheremirsec} that we can simply add a negative sign to the radius to account for these cases, so we focus our disccusion to the two cases both having their object point inside the sphere.}  in which light can refract at the spherical surfaces.

\onefig{spheresur}{scale=0.3} 

The first possibility, labelled as path 3, is that the object (at this stage we assume that it is real) is situated behind the center of curvature, which creates a real object. The second possibility, labelled as path 4, is that the object is situated closer to the surface than the center of curvature, which creates a virtual object.

However, using the sign convention defined above, we can generalized the equations governing the cases as 

\begin{equation}
    \frac{n_1 }{u} + \frac{n_2 }{v} = \frac{n_1 - n_2 }{R}.
\end{equation}

The case where the object distance \(u\) is negative, which only occurs when there are more than two imaging systems, can be regarded as if \(u\) is real but \(v \text { and } R\) are negative. Therefore, it is equivalent as if there is a real object ssituated at the opposite side of the mirror with the same distance from the mirror. This implies that we can simply treat the image point of the first imaging system as the object point of the second imaging system.

\section{Thin Lens}

From the two fundamental building blocks, we can construct a lens with two sphereical surfaces. Treating the image of the first surface as the object of the second surface, we can derive the lensmaker's equation 

\begin{equation}
    \frac{1}{f} = (n-1) \left(\frac{1}{R_1 } - \frac{1}{R_2 }\right).  
\end{equation}



Suppose our thin lens has thickness \(d(z) \ll  u,v \), which depends on height \(z\). The paraxial approximation tells us that \(z \ll  u,v\) and the path length of light travelled inside the lens is simply \(d(z)\). Therefore, using Fermat's principle, we have

\begin{equation}
    \begin{aligned} 
    L(z) &= \sqrt{\left(u-\frac{d(z)}{2}\right) ^2+z^2} + nd(z) + \sqrt{\left( v-\frac{d(z)}{2}  \right)^2+z^2} \\
    &\approx  \left( u- \frac{d(z)}{2}  \right) \left( 1 + \frac{z^2}{2\left( u - \frac{d(z)}{2}  \right)^2}  \right) + d(z) + \left( v- \frac{d(z)}{2}  \right) \left( 1 + \frac{z^2}{2\left( v - \frac{d(z)}{2}  \right)^2}  \right) \\
    &= u + v + (n-1) d(z) + \frac{z^2}{2} \left( \frac{1}{\left( u-\frac{d(z)}{2}  \right)} + \frac{1}{\left( v-\frac{d(z)}{2}  \right)}  \right) \\
    &\approx u+v+(n-1)d(z) + \frac{z^2}{2}(\frac{1}{u} + \frac{1}{v}  ) = \text{constant}.  
    \end{aligned} 
\end{equation}

Upon differentiation, 

\begin{equation}
    \begin{aligned} 
    \frac{d}{dz}d(z) &= -\frac{z}{n-1}(\frac{1}{u} + \frac{1}{v}  ) \\
    \implies  d(z) &= d(0) - \frac{z^2}{2(n-1)} \left( \frac{1}{u} + \frac{1}{v}   \right).   
    \end{aligned} 
\end{equation}

For the lens to work with arbitrary \(u, v\), we require that 

\begin{equation}
    \frac{1}{u} + \frac{1}{v} \equiv \frac{1}{f} = \text{constant}.    
\end{equation}





\section{Refracting Telescope}

\section{Copmound Microscope}



\end{document}
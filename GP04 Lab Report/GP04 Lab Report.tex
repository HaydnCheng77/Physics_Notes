\documentclass[11pt]{article}

% Packages for page layout, headers, math, figures, and hyperlinks
\usepackage{mypackage}
\usepackage[margin=1in]{geometry}
\usepackage{fancyhdr}
\usepackage{graphicx}
\usepackage{amsmath,amssymb}
\usepackage{hyperref}
\usepackage{lipsum}  % for placeholder text; remove when writing your report

% Set up the header: left header shows the lab script number, right header shows the date.
\pagestyle{fancy}
\fancyhf{asdf}
\fancyhead[L]{GP04}       % Replace GP01 with your lab script number if needed.
\fancyhead[R]{\today}      % Replace \today with a fixed date if desired.
\fancyfoot[C]{\thepage}

\title{A Study of Driven Harmonic Motion}

\begin{document}

\maketitle

% Abstract (less than 0.25 pages)
\begin{abstract}
    In this experiment, we investigated the dynamics of driven simple harmonic motion using a torsional pendulum. Firstly, we determined the effective torsional spring constant \(\kappa \)  and the natural frequency \(\omega _{0} \) of the pendulum. Then, by adding damping of different degrees to the system we measure the damping constant \(\gamma \), oscillating frequency \(\omega _{\gamma } \)  and the quality factor \(Q\). Lastly, we . Most experimental results have shown good agreement with the theoretical prediction. We are also able to provide explanations when the experimental results have shown discrepancies.
\end{abstract}

% Introduction (approximately 1 page)
\section{Introduction}

The motion of the torsional pendulum is governed by the (rotational) Newton's second law

\begin{equation}
    I \frac{d^2\theta }{dt^2}  = \tau _{D}(t) -  \gamma I \frac{d \theta }{dt} - \kappa \theta,
\end{equation}

where \(\theta \) is the angular displacement of the disc, \(I\) is the rotaional inertia of the disc, \(\gamma \) is the damping constant and \(\kappa \) is the effective torsional spring constant of the springs. The three terms on the right-hand side of the equation represent the driving torque provided by the motor \(\left( \tau _{D}(t)  \right)\), the damping torque due to the interaction with the magnet \(( - \gamma I \dot{\theta }  )\), and the restoring torque provided by the springs \((- \kappa \theta )\), respectively. 

Dividing the whole equation by \(I\) and defining the natural frequency \(\omega _{0} \) via \(\omega _{0}^2 \equiv \kappa /I \), we can rewrite the equation as 

\begin{equation}
    \frac{d^2\theta }{dt^2} + \gamma \frac{d \theta }{dt} + \omega _{0} ^2 \theta = \frac{\tau _{D}(t) }{I}. \label{main} 
\end{equation}

The homogeneous part of the differential equation can be solved by substituting the ansatz \(\theta_{H}  = e^{i \omega t} \) to find

\begin{equation}
    \omega ^2 + i \gamma \omega + \omega _{0}^2 = 0 \implies \omega = \frac{i \gamma }{2} \pm \sqrt{\omega _{0}^2 - \frac{\gamma ^2}{4}  } \equiv \frac{i \gamma }{2} \pm \omega _{\gamma }, \label{omegagamma} 
\end{equation}

where \(\omega _{\gamma } \) is the underdamped freqency, \textit{i.e.,} the frequency at which the system is oscillating if there is no driving force.

If the driving torque is sinusoidal, it can be written in the form of \(\tau _{D}(t) = \tau _{0}e^{i \omega t}   \). Then the particular solution of \(\theta (t)\) can be solved by guessing \(\theta _{P} = Ae^{i \omega t}  \), which upon substitution into \cref{main} gives

\begin{equation}
    (-\omega ^2 + i \gamma \omega +\omega _{0}^2 ) A = \frac{\tau _{0} }{I} \implies A = \frac{\tau _{0} \omega _{0}^2 }{\kappa  \sqrt{ (\omega _{0}^2-\omega ^2 )^2+\omega ^2\gamma ^2 }} e^{-i \varphi }, \quad \tan \varphi  = \frac{\omega \gamma }{\omega _{0}^2 - \omega ^2 }.  
\end{equation}

Due to the linearity of the differential equation, the genearl solution is given by the linear combination of all possible homogeneous solution plus the particular solution, which is 

\begin{equation}
    \theta (t) = e^{-\frac{\gamma }{2} } \left( A \cos (\omega t) + B\sin (\omega _{\gamma } t) \right) + \frac{\tau _{0} \omega _{0}^2 }{\kappa  \sqrt{ (\omega _{0}^2-\omega ^2 )^2+\omega ^2\gamma ^2 }} \cos (\omega t-\varphi ).
\end{equation}

Conventionally, the quality factor of an oscillating system is defined as the ratio of the natural frequecny to the damping constant, so

\begin{equation}
    Q \equiv \frac{\omega _{0} }{\gamma } \approx \frac{\omega _{\gamma } }{\gamma },
\end{equation}

where the aproximation is based on the premise that the quality factor is sufficiently high, \textit{i.e.,} \(Q \gg 1\), since if it is true then from the definition of \(\omega _{\gamma } \) from \cref{omegagamma} we can see that 

\begin{equation}
    \omega _{\gamma } = \omega _{0}\sqrt{1-\frac{1}{4Q^2} } \approx \omega _{0}.    
\end{equation}




\section{Method}


\section{Results}



\section{Interpretation}


\section{References}
\begin{itemize}
    \item Lab Script: GP04.
\end{itemize}

\end{document}
